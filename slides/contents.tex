\begin{frame}[fragile]{Überblick}
\begin{itemize}
\item Grundsätzliche Funktionsweise von Monaden
\item Beispielhafte Vorstellung mehrerer Monaden
\item Monaden in Java?!
\end{itemize}
\end{frame}


\begin{frame}[fragile]{Warum funktionale Programmierung?}
\begin{itemize}
\onslide+<2->
\item Höhere Abstraktionsmöglichkeiten
\begin{itemize}
\item z. B. Higher Order Functions, Closures
\item ermöglichen knappen, klaren Code
\end{itemize}

\onslide+<3->
\item Besseres Laufzeitverhalten
\begin{itemize}
\item durch Lazy Evaluation, d. h. Auswertung von Funktionsargumenten erst beim tatsächlichen Zugriff
\end{itemize}

\onslide+<4->
\item Parallelisierung
\begin{itemize}
\item angesichts immer größerer Prozessorzahlen zunehmend wichtiger
\end{itemize}

\onslide+<5->
\item Continuations
\begin{itemize}
\item Beschreibung des Ablaufzustands eines Programms 
\item z. B. zur Live-Migration laufender Prozesse
\end{itemize}

\onslide+<6->
\item Einfache Einbettung domänenspezifischer Sprachen
\begin{itemize}
\item inklusive Typsicherheit
\end{itemize}

\end{itemize}
\end{frame}

% Hinweis: Monaden sind eine Einbettung des imperativen Programmiermodells!

\begin{frame}[fragile]{Und was sind jetzt Monaden?}

Monaden 
\begin{itemize}
\item sind ein Abstraktionskonzept
\item vereinheitlichen unterschiedliche Aspekte
\item kapseln immer wiederkehrende Aspekte
\item helfen dadurch beim Fokussieren
\item schlagen (ganz nebenbei) eine Brücke zwischen der funktionalen und der nichtfunktionalen Welt
\end{itemize}

\end{frame}


\begin{frame}[fragile]{Drei einfache Funktionen}
\begin{lstlisting}
    Integer minus5(Integer x) {
        if (x == 7) {
            return null;
        }
        return x - 5;
    }

    Integer mal3(Integer x) {
        if (x % 2 == 0) {
            return null;
        }
        return x * 3;
    }

    Integer plus7(Integer x) {
        return x + 7;
    }
\end{lstlisting}
\end{frame}

\begin{frame}[fragile]{Die Funktionen hintereinandergeschaltet}
\begin{lstlisting}
minus5 (mal3 (plus7 ( 12 )))
\end{lstlisting}

liefert 52

\onslide+<2->

\vspace{3em}

\begin{lstlisting}
minus5 (mal3 (plus7 ( 1 )))
\end{lstlisting}

liefert eine Exception...

\onslide+<3->
~

(wir erinnern uns: \lstinline|mal3| mag nur ungerade Zahlen)

\end{frame}

\begin{frame}[fragile]{Also entweder besser aufpassen...}

\begin{lstlisting}
    Integer aufpassen(Integer x1) {
        if (x1 != null) {
            Integer x2 = plus7(x1);
            if (x2 != null) {
                Integer x3 = mal3(x2);
                if (x3 != null) {
                    return minus5(x3);
                }
            }
        }
        return null;
    }
\end{lstlisting}


\end{frame}

\begin{frame}[fragile]{...oder robustere Funktionen bauen?}
\begin{lstlisting}
    Integer minus5(Integer x) {
        if (x == null || x == 7) {
            return null;
        }
        return x - 5;
    }

    Integer mal3(Integer x) {
        if (x == null || x % 2 == 0) {
            return null;
        }
        return x * 3;
    }

    Integer plus7(Integer x) {
        if (x == null) { return null; }
        return x + 7;
    }
\end{lstlisting}
\end{frame}

\begin{frame}[fragile]{Der Effekt}
\begin{itemize}
\item Die null-Prüfungen müssen überall eingefügt werden
\item Vergessene Prüfungen führen zu Laufzeitfehlern
\item Der Prüf-Code lenkt von der eigentlichen Fachlichkeit ab
\end{itemize}
\end{frame}


\begin{frame}[fragile]{Die drei Funktionen in Haskell}

Das Haskell-Äquivalent zu nullable Types:

\begin{lstlisting}
data Maybe a = Just a
             | Nothing
\end{lstlisting}

\onslide+<2->

Unsere Funktionen in Haskell:

\begin{lstlisting}
minus5 Nothing = Nothing
minus5 (Just x)
    | x == 7    = Nothing
    | otherwise = Just (x - 5)

mal3 Nothing = Nothing
mal3 (Just x)
    | even x    = Nothing
    | otherwise = Just (3 * x)

plus7 Nothing = Nothing
plus7 (Just x) = Just (x + 7)
\end{lstlisting}
\end{frame}

\begin{frame}[fragile]{Separation of Concerns}

Unsere Funktionen sollen sich nur um die Fachlichkeit kümmern:

\begin{lstlisting}
minus5 x
    | x == 7    = Nothing
    | otherwise = Just (x - 5)

mal3 x
    | even x    = Nothing
    | otherwise = Just (3 * x)

plus7 x = Just (x + 7)
\end{lstlisting}
\end{frame}

\begin{frame}[fragile]{Auslagern der Null-Behandlung}

Das Verbinden der Funktionen lagern wir aus:
\begin{lstlisting}
(>>=) :: Maybe a -> (a -> Maybe b) -> Maybe b

Nothing  >>= f  =  Nothing
(Just x) >>= f  =  f x
\end{lstlisting}

\onslide+<2->
~

Der initiale Einstieg in den Maybe-Typ:

\begin{lstlisting}
return :: a -> Maybe a

return x  =  Just x
\end{lstlisting}

\onslide+<3->
~

Unsere Methodenaufrufe:

\begin{lstlisting}
return 12 >>= plus7 >>= mal3 >>= minus5
\end{lstlisting}

\end{frame}


\begin{frame}[fragile]{Wann kommen denn endlich die Monaden?}

\onslide+<2->

Einen Datentyp, zum Beispiel

\begin{lstlisting}
data Maybe a = Just a
             | Nothing
\end{lstlisting}

\onslide+<3->
~

auf dem zwei Funktionen

\begin{lstlisting}
(>>=) :: Maybe a -> (a -> Maybe b) -> Maybe b

return :: a -> Maybe a
\end{lstlisting}

definiert sind,

\onslide+<4->
~

nennt man eine Monade! $^\ast$

\end{frame}

\begin{frame}[fragile]{Vorteile und Nachteile}
Vorteile von Monaden:
\begin{itemize}
\item Boilerplate Code ist wegabstrahiert
\item Höheres konzeptionelles Niveau des Codes
\item Es gibt Funktionsbibliotheken, die auf Monaden aufbauen
\item Verschiedene Monaden lassen sich miteinander kombinieren (Monaden-Transformer)
\end{itemize}

\onslide+<2->
~

Nachteile von Monaden:
\begin{itemize}
\item Größere Einstiegshürde und Lernkurve
\item Benötigt implizites Wissen über die Funktionsweise
\end{itemize}

\end{frame}


%%%%%%%%%%%%%%%%%%%%%%%%%%%%%%%%%%%%%%%%%%%%%%%%%%%%



\begin{frame}[fragile]{Die einfachste Monade}

Der Datentyp:
\begin{lstlisting}
a
\end{lstlisting}

Die Funktionen:
\begin{lstlisting}
(>>=) :: a -> (a -> b) -> b

x >>= f = f x
\end{lstlisting}


\begin{lstlisting}
return :: a -> a

return x = x
\end{lstlisting}

\end{frame}

\begin{frame}[fragile]{Exception-Handling}

\begin{lstlisting}
public int throwAt7(int number){
    if( number == 7 ){
        throw new IllegalArgumentException("Ich hasse 7!");
    }
    
    return number + 1;
}
\end{lstlisting}

~

In rein funktionalen Sprachen ist dies so nicht möglich!

\end{frame}


\begin{frame}[fragile]{Eine Monade als Alternative zu Exceptions}

Der Datentyp:
\lstinputlisting[linerange={4-7}]{../haskell/Exception.hs}

enthält entweder eine Exception oder die Rückgabe.

\onslide+<2->

Die Funktionen:
\begin{lstlisting}
(>>=) :: Exceptional e s -> (s -> Exceptional e s) -> Exceptional e s

Exception exc >>= _   =  Exception exc
Success x     >>= f   =  f x
\end{lstlisting}


\begin{lstlisting}
return :: s -> Exceptional e s

return = Success
\end{lstlisting}
\end{frame}

\begin{frame}[fragile]{throw und catch}
\begin{lstlisting}
throw :: e -> Exceptional e s

throw  =  Exception
\end{lstlisting}

~

\begin{lstlisting}
catch :: Exceptional e s -> (e -> Exceptional e s) -> Exceptional e s

catch (Exception  exc) f  =  f exc
catch (Success x)      _  =  Success x
\end{lstlisting}

\end{frame}


\begin{frame}[fragile]{Anwendungsbeispiel der Exception-Monade}
\begin{lstlisting}
public int throwAt7(int x){
    if( x == 7 ){
        throw new IllegalArgumentException("Ich hasse 7!");
    }
    
    return x + 1;
}
\end{lstlisting}

\onslide+<2->

\begin{lstlisting}
data Exceptions = IllegalArgumentException String

throwAt7 x
   | x == 7     =  throw (IllegalArgumentException "Ich hasse 7!")
   | otherwise  =  Success (x + 1)
\end{lstlisting}


\end{frame}

\begin{frame}[fragile]{Funktionsverknüpfung in der Exception-Monade}
\begin{lstlisting}
throwAt7 x
   | x == 7     =  throw (IllegalArgumentException "Ich hasse 7!")
   | otherwise  =  Success (x + 1)
\end{lstlisting}

\begin{lstlisting}
mal3 :: Int -> Int
mal3 x  =  3 * x
\end{lstlisting}

\onslide+<2->

Herstellen der passenden Signatur für mal3:

\begin{lstlisting}
return . mal3
\end{lstlisting}

hat die Signatur

\begin{lstlisting}
Int -> Exceptional e Int
\end{lstlisting}

\end{frame}


\begin{frame}[fragile]{Funktionsverknüpfung in der Exception-Monade}

\lstinputlisting[linerange={43-43}]{../haskell/Exception.hs}
liefert \lstinline|Success 39|

\onslide+<3->

\lstinputlisting[linerange={42-42}]{../haskell/Exception.hs}
liefert \lstinline|Exception (IllegalArgumentException "Ich hasse 7!")|



\end{frame}

\begin{frame}[fragile]{Vorteile}

\begin{itemize}
\item Sprachunterstützung für Exceptions ist nicht erforderlich
\item Keine versehentlichen Programmabbrüche durch Exceptions
\item Volle Kontrolle über den Programmfluss, trotzdem kein Boilerplate-Code
\end{itemize}

\end{frame}


\begin{frame}[fragile]{System Output}

\begin{lstlisting}
public int numberAndText(){

    System.out.println("Hallo Herbstcampus!");
    
    return 7;
}
\end{lstlisting}

\end{frame}


\begin{frame}[fragile]{Eine Monade als Alternative zu System Output}

Der Datentyp:
\begin{lstlisting}
data Sysout a = ResOut a String
        deriving (Show)
\end{lstlisting}

soll sämtliche Systemausgaben aufsammeln.

\onslide+<2->
~

Die Funktionen:
\begin{lstlisting}
(>>=) :: Sysout a -> (a -> Sysout b) -> Sysout b

(ResOut res1 out1) >>= f = let (ResOut res2 out2) = f res1
                           in  (ResOut res2 (out1 ++ out2))
\end{lstlisting}


\begin{lstlisting}
return :: a -> Sysout a

return x = ResOut x ""
\end{lstlisting}


\end{frame}



\begin{frame}[fragile]{Anwendungsbeispiel}
\begin{lstlisting}
public int numberAndText(){

    System.out.println("Hallo Herbstcampus!");
    
    return 7;
}
\end{lstlisting}

\begin{lstlisting}
numberAndText =	ResOut 7 "Hallo Herbstcampus!"
\end{lstlisting}

\onslide+<2->

\begin{lstlisting}
public int calcAndText(int x){

    System.out.println("Ich kann rechnen!");
    
    return 4 + x;
}
\end{lstlisting}

\begin{lstlisting}
calcAndText x =	ResOut (4 + x) "Ich kann rechnen!"
\end{lstlisting}

\end{frame}


\begin{frame}[fragile]{Funktionsverknüpfung}

\begin{lstlisting}
print (numberAndText >>= calcAndText)
\end{lstlisting}
liefert \lstinline|ResOut 11 "Hallo Herbstcampus! Ich kann rechnen!"|

\end{frame}


\begin{frame}[fragile]{Vorteile}

\begin{itemize}
\item System Output bleibt stets in der richtigen Reihenfolge, auch bei Parallelisierung
\item Die tatsächliche Textausgabe ist nicht mehr über den ganzen Code verteilt, sondern zentral gebündelt
\end{itemize}

\end{frame}


%%%%%%%%%%%%%%%%%%%%%%%%%%%%%%%%%%%%%%%%%%%%%%%%%%%%


\begin{frame}[fragile]{Grundsätzliches}

\begin{itemize}

\item Es gibt keine Seiteneffekte $\Rightarrow$ Funktionen problemlos testbar

\item Man kann sofort erkennen, ob eine Funktion eine bestimmte Monade einsetzt (z. B. einen bestimmten Seiteneffekt hat)

\item Manche Monaden garantieren, dass alle Ausdrücke nacheinander ausgeführt werden, keine Vertauschung der Ausführungsreihenfolge durch lazy evaluation

\end{itemize}

%früher: explizite Fehlerbehandlung (return 0 bzw. <> 0, jeweils explizite Prüfung und Propagation)
%heute: es gibt ein implizites Modell, nämlich Exceptions
%Zukunft: Wir wollen viele solche Modelle nebeneinander haben können
%wir wollen die Modelle auch mischen können

\end{frame}


\begin{frame}[fragile]{Monaden in Java}

Die Maybe-Monade:

\begin{lstlisting}
Nothing  >>= f  =  Nothing
(Just x) >>= f  =  f x

return x  =  Just x
\end{lstlisting}

\onslide+<2->

wird für den Maybe-Typ \lstinline|Integer| zu

\begin{lstlisting}
interface Func {
    Integer eval(int arg);
}
\end{lstlisting}

\begin{lstlisting}
Integer bind(Integer arg, Func f) {
    if( arg == null ) return null;
    return f.eval(arg.intValue());
}

Integer _return( int arg ){   return Integer.valueOf(arg);   }
\end{lstlisting}
\end{frame}


\begin{frame}[fragile]{Monadische Funktionsverknüpfung in Java}
Aus
\begin{lstlisting}
return 12 >>= plus7 >>= mal3 >>= minus5
\end{lstlisting}

wird
\end{frame}

\begin{frame}[fragile]{Monadische Funktionsverknüpfung in Java}
\begin{lstlisting}
		bind(bind(bind(_return(12), new Func() {

			@Override
			public Integer eval(int arg) {
				return plus7(arg);
			}
		}), new Func() {

			@Override
			public Integer eval(int arg) {
				return mal3(arg);
			}
		}), new Func() {

			@Override
			public Integer eval(int arg) {
				return minus5(arg);
			}
		});
\end{lstlisting}
\end{frame}


%------



%%%%%%%%%%%%%%%%%%%%%%%%%%%%%%%%%%%%%%%%%%%%%%%%%%
{
%\usebackgroundtemplate{\includegraphics[width=\paperwidth,height=\paperheight]{background-slide.png}}
\begin{frame}{Vielen Dank!}

        Code \& Folien auf GitHub:
        \begin{center}
                \url{https://github.com/NicoleRauch/Monaden}
        \end{center}

        \begin{block}{Nicole Rauch}
        \begin{description}[Twitterxx]
                \item[E-Mail]  \href{mailto:nicole.rauch@msg-gillardon.de}{\texttt{nicole.rauch@msg-gillardon.de}}
                \item[Twitter] \href{http://twitter.com/NicoleRauch}{\texttt{@NicoleRauch}}
        \end{description}
        \end{block}
\end{frame}
}
